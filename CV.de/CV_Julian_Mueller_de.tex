\documentclass[noindent]{scrartcl}
\usepackage{../Header/header}

\usepackage[a4paper,width=13cm, height=27cm]{geometry}

\setlist{nosep}

\begin{document}

\newcommand{\snp}{\href{www.snpgroup.com}{SNP SE}}
\newcommand{\herrenknecht}{\href{https://www.herrenknecht.com/}{Herrenknecht AG}}
\newcommand{\sdte}{\href{https://news.sap.com/germany/2019/05/arbeitsgruppe-migration-s4hana/}{Selective Data Transition Engagement}}

%----------------------------------------------------------------------------------------
%	NAME AND CONTACT INFORMATION SECTION
%----------------------------------------------------------------------------------------

\thispagestyle{empty}
\begin{tikzpicture}[remember picture,overlay]
    \node[align=center] at ([yshift=-7cm]current page.north)%
         {\huge \spacedallcaps{Curriculum Vitae}};
    % cover photo
    \node[align=center] (image) at ([yshift=-13cm]current page.north) %
         {\includegraphics[width=75mm]{Foto}};
    % name and address
    \node[align=center] (name) at ([yshift=-19cm]current page.north) %
         {\Huge \textcolor{magma}{\spacedallcaps{Julian Müller}}};
    \node[align=center] at ([yshift=-20.2cm]current page.north) %
         {\textcolor{magma}{\spacedallcaps{Maschinenbau, (B.E.)}}};
    \node[align=center] at ([yshift=-21.2cm]current page.north) %
         {\textcolor{magma}{\spacedallcaps{Logik, (M.A.)}}};
    \node[align=left] at ([xshift=0.3cm,yshift=-25cm]current page.north) % 
         {\color{textgray}
			\begin{tabular}{lr}
             Geburtsdatum & 29 Oktober 1984 \vspace{0.5em} \\

             Geburtsort & 77694 Kehl \vspace{0.5em} \\

             Email & \href{mailto:jul.mue@hotmail.de}{jul.mue@hotmail.de} \vspace{0.5em} \\

             % Website & \href{https://julmue.github.io}{julmue.github.io} \vspace{0.5em} \\

             Telefon & +49 176 20562999 \vspace{0.5em}  \\

             Adresse & Schlo\ss stra\ss e 34 \vspace{0.5em}  \\
                       & 14059 Berlin \\
            \end{tabular}
         };
\end{tikzpicture}
%\thispagestyle{empty} % Stop the page count at the bottom of the first page

% Your name
\newpage

\thispagestyle{empty}

\begin{cv}
%\begin{cv}{\spacedallcaps{Julian Müller}}\vspace{1.5em} 

%----------------------------------------------------------------------------------------
%	WORK EXPERIENCE
%----------------------------------------------------------------------------------------

\topic{Berufspraxis}

% ------------------------------------------------------------------------------
% ------------------------------------------------------------------------------
% ------------------------------------------------------------------------------
\NewEntry{seit 2020/01}{Engineering Manager / Solution Owner R\&D}{\snp}

\Description{\MarginText{Engineering Manager / Solution Owner R\&D}%
\highlight{Design und Umsetzung eines Produkts zur Prozessautomatisierung} basierend auf symbolischer KI, 
funktionaler Programmierung, ABAP und HANA.
\highlight{Aufbau und Führung eines verteilten High-Preformance Teams} zur Umsetzung des Produkts.
\begin{itemize}[label=\isym]
\item \textbf{Engineering Manager}: Aufbau und Führung eines 5-köpfigen verteilten Entwicklungsteams (Teamsprache: Englisch)
\item \textbf{Solution Owner}: Agiler Produktentwurf und Entwicklungssteuerung
\item \textbf{Lead Engineer}: Systementwurf, Architektur, Implementierung von Prototypen
\end{itemize}
Weitere Funktion
\begin{itemize}[label=\isym]
\item \textbf{Lead Engineer} für die SNP im \highlight{\sdte\ }der SAP
\end{itemize}
}

\NewEntry{01/2019-01/2020}{Engineer SAP Transformation R\&D}{\snp}

\Description{\MarginText{Engineer R\&D}%
\highlight{Entwicklung von Systemen für die selektive Datenmigration nach S/4 HANA}:
\begin{itemize}[label=\isym]
\item \highlight{Automatisierung der Finanz- und Logistikmigration nach S/4 HANA}
\item Entwicklung von \highlight{Analyse- und Tracingsystemen}
\item \highlight{Data Engineering} und \highlight{Data Analysis}
\item Presales, Schulungen und Vorträge
\end{itemize}
Weiter:
\begin{itemize}[label=\isym]
\item Absolvent des \highlight{SNP Exzellenzprogramms für junge Führungskräfte}
\end{itemize}
}


\NewEntry{01/2017-01/2019}{Consultant SAP Transformation}{\snp}

\Description{\MarginText{Consultant SAP Transformation}%
\highlight{Entwicklung von Systemen für die  SAP Landscape Transformation (ERP und BW)}:
\begin{itemize}[label=\isym]
\item Entwicklung von \highlight{Systemen für die SAP Business Warehouse Transformation}
\item Transformation SAP Business Warehouse in Projekten
\item \highlight{Data Engineering} und \highlight{Data Analysis}
\item Presales, Schulungen und Vorträge
\end{itemize}
}


\NewEntry{10/2010-07/2012}{Ingenieur R\&D (Werkstud.)}{\herrenknecht}

\Description{\MarginText{Ingenieur R\&D (Werkstudent)}%
\highlight{Anwendung von Verfahren der künstlichen Intelligenz} zur Prozesssicherung im Tunnelvortrieb.
\highlight{Entwurf, Programmierung und Inbetriebnahme eines 3D-Lasersystems} zur Verleißmessung bei Abbauwerkzeugen.
\begin{itemize}[label=\isym]
% \item Berechnung der optischen Komponenten und der Antriebseinheit
\item Planung, Realisierung und Inbetriebnahme des Prototypen
\item Test und Untersuchung von Verfahren zur Tiefenbildgewinnung
\item Untersuchung von Verfahren zur Merkmalsextraktion in Grauwertbildern
\item Anforderungsermittlung und Konzeption des Messsystems
\end{itemize}
}

\NewEntry{09/2010-03/2011}{Ingenieur R\&D (Praktikum)}{\herrenknecht}

\Description{\MarginText{Ingenieur R\&D (Praktikum)}%
Entwicklung von Konzepten zum maschinellen Tunnelvortrieb.
\begin{itemize}[label=\isym]
\item Projekt: maschineller Tunnelvortrieb in der nuklearen Forschung
\item Maschinenkonzept: Aufweitung bestehender Tunnel unter Aufrechterhaltung des Straßen- und Bahnverkehrs
\end{itemize}
}

\NewEntry{01/2009-09/2009}{Tutor (Mathematik)}{RFH Köln}

\Description{\MarginText{Tutor\\Mathematik}%
Tutorium für Ingenieursmathematik.
\begin{itemize}[label=\isym]
\item Grundlagen der Ingenieursmathematik
\item Differenzial-/Integralrechnung
\item Vektorrechnung
\end{itemize}
}


%----------------------------------------------------------------------------------------
%	EDUCATION
%----------------------------------------------------------------------------------------
\newpage

\topic{Ausbildung}

\NewEntry{2013-2016}{Logik (M.A.)}{Universität Leipzig}

\Description{\MarginText{\href{http://www.sozphil.uni-leipzig.de/cm/logik/}{Master Logik}\\
(Endnote: 1.6)}%
%Abschluss: 1.6 (angestrebt)\newline%
Spezialisierungen:
\EvenThree
    {\isym\ \textit{Constraint-Programmierung}}%
    {\isym\ \textit{Parakonsistente Logiken}}%
    {\isym\ \textit{Wissensrepräsentation}}
{Abschlussarbeit (Note 1.0): \textit{Das untypisierte Lambdakalkül und seine Anwendung}\newline
\isym\ \textit{In der Informatik}: Als Grundlage funktionaler Programmiersprachen\newline
\isym\ \textit{In der Beweistheorie}: Curry-Howard-Lambek-Isomorphismus\newline
\isym\ \textit{In der Mathematik}: Als interne Sprache kartesisch geschlossener Kategorien\newline
Betreuer: Dr.~Peter \textsc{Steinacker}, Prof.~Thomas \textsc{Bartelborth}}\newline
(Note 1.0)}

%------------------------------------------------

\NewEntry{2008-2013}{Maschinenbau (B.E.)}{RFH Köln}

\Description{\MarginText{Bachelor Maschinenbau\\(Endnote: 1.8)}%
Spezialisierungen:
\EvenThree%
    {\isym\ \textit{Technische Optik / Lasertechnik}}%
    {\isym\ \textit{Mechatronik}}%
    {\isym\ \textit{Programmieren in C}}%
Abschlussarbeit (Note 1.0): \textit{Verschlei\ss erkennung bei Werkzeugen im Tunnelbau}\newline
\isym\ \textit{Entwicklung, Bau und Inbetriebnahme eines Laserscanners zur Verschlei\ss messung}\newline
\isym\ \textit{Vergleich von Verfahren der industriellen Bildverarbeitung (2d/3d)}\newline
\isym\ \textit{Projektspezifische Analyse geometrischer und logistischer Randbedingungen von\\ 
               \hspace{0.8em}Vortriebsprozessen im maschinellen Tunnelbau}\newline
Betreuer: Prof.~Werner \textsc{Simon}, Prof.~Marcus \textsc{Scholl}}

\NewEntry{2004-2007}{Event Organizer}{Angell Institut Freiburg}

\Description{\MarginText{Event Organizer}%
Ausbildung zum \textit{International Event Organizer}.}


%----------------------------------------------------------------------------------------
%	METHODOLOGY
%----------------------------------------------------------------------------------------
\topic{Methoden und Kenntnisse}

\NewMethodology{Team Development und Leadership}
\Description{\MarginText{Team Development und Leadership}

\begin{itemize}[label=\isym]
\item Aufbau und Führung verteilter Entwicklungsteams (Teamsprache Englisch)
\item Auswahl, Recruiting und Onboarding von Teammitgliedern
\item Coaching in Kommunikations- und Feedbackkultur
\item Coaching in Methoden des Softwareentwurfs und der Softwarequalität
\item Workshops zu Team- und Unternehmenswerte (The golden Circle - What, How, Why)
\item Einführung und Verbesserung agiler Entwicklungsmethoden
\item Integration interner und externer Stakeholder
\end{itemize}
}


\NewMethodology{Agile Entwicklungsmethoden}
\Description{\MarginText{Agile Entwicklungsmethoden}

\begin{itemize}[label=\isym]
\item Scrum und Kanban (aktuelle Rolle: Solution Owner)
\item Design Thinking (Empthy, Define, Ideate)
\item Lean Management und Lean Startup
\end{itemize}
}

\NewMethodology{Projekterfahrung}
\Description{\MarginText{Projekterfahrung}

\begin{itemize}[label=\isym]
\item Engineering Manager - Aufbau und Führung eines Entwicklungsteams \details{proj_engineering_manager}
\item Solution Owner - Process Automation \details{proj_solution_owner}
\item Lead Engineer - SAP Selective Data Transition Engagement \details{proj_sdte}
\item Project Lead - Entwicklung einer Roadmap nach S/4 HANA \details{proj_roadmap_s4}
\item Engineer - Migration Rechnungswesen und Logistik nach S/4 HANA \details{proj_trafo_s4}
\item Engineer - Transformation SAP BW \details{proj_trafo_bw}
\item Trainer - Schulungsleitung ABAP Programming for SAP HANA \details{proj_training_s4}
\end{itemize}
}


\NewMethodology{Software Engineering}
\Description{\MarginText{Software Engineering}%

\begin{itemize}[label=\isym]
\item SAP Technologien \details{tec_sap}
\item SAP Landscape Transformation \details{tec_landscape_transformation}
\item Künstliche Intelligenz \details{tec_ai}
\item Data Science und Data Engineering  \details{tec_data_science}
\item Compilerbau und Entwicklung von Programmiersprachen \details{tec_language}
\item Infrastruktur, Continuous Integration und Deployment \details{tec_infrastructure}
\item Programmiersprachen \details{tec_languages}
\item Eigene Softwareprojekte \details{software_projects}
\end{itemize}
}


%----------------------------------------------------------------------------------------
%	TECHNOLOGY
%----------------------------------------------------------------------------------------
\newpage

\topic{Technologien}

\NewTechnology{SAP Technologien}
\label{tec_sap}
\Description{\MarginText{SAP Technologien}

\begin{itemize}[label=\isym]
\item SAP ABAP: ABAP OO, AMDPs, CDS-Views, HANA-Proxy-Objects, \dots
\item SAP HANA: HANA SQL, Core Data Services, SQLScript, \dots
\item SAP ERP: S/4 HANA, R3
\item SAP BW
\end{itemize}
}


\NewTechnology{SAP Landscape Transformation}
\label{tec_landscape_transformation}

\Description{\MarginText{SAP Landscape Transformation}%
\highlight{SAP Transformation Solutions}
\begin{itemize}[label=\isym]
\item Finance Conversion to S/4 HANA
\item Logistics Conversion to S/4 HANA
\item Conversion Methods: XPRA, XCLA, SDM, DDIC Conversion Exits
\end{itemize}
\highlight{SNP Transformation Solutions}
\begin{itemize}[label=\isym]
\item SNP Mission Control (Prozesskontrolle der Transformation)
\item SNP S/4 Cockpit (Migration nach S/4 HANA)
\item SNP Transformation Cockpit (SAP Landscape Transformation Solution)
\end{itemize}
}


\NewTechnology{Künstliche Intelligenz}
\label{tec_ai}

\Description{\MarginText{SAP Landscape Transformation}%
Verfahren symbolischer künstlicher Intelligenz
\begin{itemize}[label=\isym]
\item \href{https://en.wikipedia.org/wiki/Automated_planning_and_scheduling}{Automatisierte Planung}
\item \href{https://en.wikipedia.org/wiki/Boolean_satisfiability_problem}{Propositional Satisfiability (SAT)}
\item \href{https://en.wikipedia.org/wiki/Satisfiability_modulo_theories}{Satisfiability Modulo Theories (SMT)}
\item \href{https://en.wikipedia.org/wiki/Model_checking}{Bounded Model Checking}
\end{itemize}
Verfahren numerischer künstlicher Intelligenz
\begin{itemize}[label=\isym]
\item \href{https://en.wikipedia.org/wiki/Artificial_neural_network}{Künstliche Neuronale Netzwerke (KNN)}
\item \href{https://en.wikipedia.org/wiki/Support_vector_machine}{Support-Vector-Machine (SVM)}
\end{itemize}
}


\NewTechnology{Data Science und Data Engineering}
\label{tec_data_science}

\Description{\MarginText{Data Science und Data Engineering}%
Python, Haskell und R im Frontend sowie ABAP und HANA im Backend.
\begin{itemize}[label=\isym]
\item Datenaufbereitung in R und Pyhton
\item Visualisierung und Dashboards: Jupyter, Plotly, Seaborn
\item Datenbankanbindung mit SQL und ODBC
\end{itemize}
}


\NewTechnology{Compilerbau und Entwicklung von Programmiersprachen}
\label{tec_language}

\Description{\MarginText{Entwicklung von Programmiersprachen}%
Entwicklung von funktionalen und imperativen Programmiersprachen in Haskell.
\begin{itemize}[label=\isym]
\item Grammatiken und Produktionsregeln
\item Typensysteme 
\item Parser und Prettyprinter
\item Interpreter und Transpiler
\end{itemize}
}


\NewTechnology{Infrastruktur, Continuous Integration und Deployment}
\label{tec_infrastructure}
\Description{\MarginText{Infrastruktur}

\begin{itemize}[label=\isym]
\item Virtualisierung: Hyper-V, Virtualbox
\item Versionierung und Continuous Integration: Git, Gitlab, abapGit
\item Containerisierung: Docker
\item Dependency Management: Make, Nix
\end{itemize}
}

\NewTechnology{Programmiersprachen}
\label{tec_languages}
\Description{\MarginText{Programmiersprachen}

\begin{itemize}[label=\isym]
\item ABAP
\item Haskell
\item Python
\item R
\item SQL, SQLScript
\item Prolog
\item C
\end{itemize}
}





%----------------------------------------------------------------------------------------
%	Projects
%----------------------------------------------------------------------------------------
\newpage

\topic{Projekterfahrung}

\NewProject{Seit 2020/01}{Aufbau und Führung eines Dev-Teams}{SNP}
\label{proj_engineering_manager}

\Description{\MarginText{\highlight{Engineering Manager}}%
%\textbf{Beschreibung}\newline
Aufbau und Führung eines verteilten 5-köpfigen agilen Entwicklungsteams; die Teamsprache ist Englisch.
Schwerpunkte sind SAP ABAP und SAP HANA, Compilerbau, funktionale Programmierung sowie Verfahren der symbolischen künstlichen Intelligenz.\newline
\textbf{Tätigkeiten}
\begin{itemize}[label=\isym]
\item Recruiting und Onboarding
\item Coaching und Teamentwicklung
\item Implementieren agiler Entwicklungsmethoden
\end{itemize}
\textbf{Technologien (Auszug)}
\begin{itemize}[label=\isym]
\item Jira, Confluence
\item Microsoft Teams, MS Office
\item Mural
\end{itemize}
}


\NewProject{Seit 2020/01}{Produktentwicklung Process Automation}{SNP}
\label{proj_solution_owner}

\Description{\MarginText{\highlight{Solution Owner}}%
%\textbf{Beschreibung}\newline
Design und Umsetzung eines Produkts im Bereich Prozessautomatisierung basierend auf symbolischer KI, funktionaler Programmierung, ABAP und HANA. 
Meine Tätigkeiten umfassen dabei die Aufgaben des Solution Owners als auch die Aufgaben des Lead Engineers.\newline
\textbf{Tätigkeiten}
\begin{itemize}[label=\isym]
\item Konzeption der Produktvision
\item Definition der Roadmap
\item Kommunikation und Abstimmung mit Stakeholdern und Experten
\item Pflege und Priorisierung des Backlogs
\item Sprintplanung und Scrumevents
\item Implementierung von Prototypen
\item Koordination der Forschung
\end{itemize}
\textbf{Technologien (Auszug)}
\begin{itemize}[label=\isym]
\item Verfahren und Systeme der symbolischen künstlichen Intelligenz
\item Verfahren und Systeme des Compilerbaus und der funktionalen Programmierung
\item SAP ABAP
\item SAP HANA (Core Data Services, Graph Database, SQLScript)
\end{itemize}
}


\NewProject{Seit 2019/02}{SAP Selective Data Transition Engagement}{SAP}
\label{proj_sdte}

\Description{\MarginText{\highlight{Lead Engineer}}%
%\textbf{Beschreibung}\newline
Das \sdte\ ist eine Arbeitsgruppe aus SAP, SNP, CBS, 
Natuvion und Datavard in welchem Konzepte, Technologien und Best-Practices erarbeitet werden um den Top 500 der ERP-Großkunden die Migration auf SAP S/4HANA zu ermöglichen.\newline
\textbf{Tätigkeiten}
\begin{itemize}[label=\isym]
\item Leitender Vertreter der SNP im Technology Stream des SDTE Projekts
\item Koordination mit anderen Streams (Governance, Architecture, Application)
\item Entwicklung einer graphenbasierten Wissensdatenbank zur Analyse von SAP-Konvertierungsmethoden (XPRA, XCLA, SDM, DDIC-Conversion-Exit)
     \begin{itemize}[label=\cdot]
	\item Anforderungen und Konzept
	\item Prototypenbau und Proof-Of-Concept
	\item Solution Owner für die Implementierung
     \end{itemize}
\end{itemize}	
\textbf{Technologien}
\begin{itemize}[label=\isym]
\item SAP Konvertierungstechniken: XPRA, XCLA, SDM, DDIC-Conversion-Exit
\item SAP HANA (Graph Database, SQLScript)
\item SAP ABAP
\end{itemize}
}

% Seitenumbruch
\newpage

\NewProject{2019/01-2019/10}{Roadmap nach S/4 HANA}{SNP}
\label{proj_roadmap_s4}

\Description{\MarginText{\highlight{Project Lead}}%
%\textbf{Beschreibung}\newline
Im Rahmen des Excellenzclusters wurde eine Roadmap nach S/4 HANA entwickelt. 
Basierend auf dem Design-Thinking-Prozess wurden in enger Zusammenarbeit 
mit den Stakeholdern diverse funktionale Prototypen erstellt.\newline
\textbf{Tätigkeiten}
\begin{itemize}[label=\isym]
\item Design Thinking: Planung und Durchführung von Workshops mit Wissensträgern und Stakeholdern
\item Design Thinking: Erstellen von Personas
\item Design Thinking: Entwicklung funktionaler Prototypen
\end{itemize}
\textbf{Technologien}
\begin{itemize}[label=\isym]
\item SAP ABAP
\item SAP HANA
\item Verfahren der symbolischen künstlichen Intelligenz
\item Verfahren des Compilerbaus und der funktionalen Programmierung
\end{itemize}
}


\NewProject{Seit 2019/01}{Migration nach S/4 HANA}{SNP}
\label{proj_trafo_s4}

\Description{\MarginText{\highlight{Engineer}}%
%\textbf{Beschreibung}\newline
Der Aufbau und die Integration des Rechnungswesens und der Logistik 
sind kritische Elemente für eine Migration von R3 nach S/4 HANA.\newline
\textbf{Tätigkeiten}
\begin{itemize}[label=\isym]
\item Integration von Migrationslogik in das SNP S/4 Cockpit
\item System- und Fehleranalyse
\item Projektsupport 
\item Ausarbeitung und Durchführung von Schulungen zur S/4 HANA Migration
\end{itemize}
\textbf{Technologien}
\begin{itemize}[label=\isym]
\item SAP ABAP
\item SAP HANA
\item SNP S/4 Cockpit
\end{itemize}
}


\NewProject{2017/07-2018/07}{Konversion SAP BW}{Bosch, SNP}
\label{proj_trafo_bw}

\Description{\MarginText{\highlight{Engineer}}%
%\textbf{Beschreibung}\newline
Konversion einer aus fünf Systemen bestehenden BW-Landschaft basierend auf dem 
Regelwerk einer sich gleichzeitig in der Entwicklung befindlichen ERP-Migration.\newline
\textbf{Tätigkeiten}
\begin{itemize}[label=\isym]
\item Technische Projektleitung eines 6-löpfigen Teams
\item Konzept und Umsetzung: Migration DSO InfoProvider
     \begin{itemize}[label=\cdot]
	\item Prototypenbau und Proof-Of-Concept
     \item Implementierung Full-Scope im Projekt
     \end{itemize}
\item Softwareentwicklung: Analysegetriebene Migration
     \begin{itemize}[label=\cdot]
	\item Analysebasierte Konfiguration der SNP Transformationswerkzeuge
	\item Sprachen: ABAP, R Language, Haskell
     \end{itemize}
\end{itemize}
\textbf{Technologien}
\begin{itemize}[label=\isym]
\item SAP BW
\item SNP TC
\item R Language
\item Haskell
\end{itemize}
}


\NewProject{2017/07}{Schulungsleitung ABAP Programming for SAP HANA}{Mercoline, SNP}
\label{proj_training_s4}

\Description{\MarginText{\highlight{Trainer}}%
%\textbf{Beschreibung}\newline
Vorbereitung und Leitung von Schulungen zum Thema ABAP für HANA, ABAP in Eclipse und modernes ABAP (ABAP Version >= 7.49)
\textbf{Tätigkeiten}
\begin{itemize}[label=\isym]
\item Ausarbeitung Schulungsunterlagen und Schulungsübungen
\item Durchführung zweier einwöchiger Schulungen beim Kunden (jeweils 8 Schulungsteilnehmer)
\item Inhalte entsprechen HA400 (ABAP Programming in HANA) und BC404 (ABAP in Eclipse)
\end{itemize}
\textbf{Technologien}
\begin{itemize}[label=\isym]
\item ABAP
\item HANA
\end{itemize}
}


% ------------------------------------------------------------------------------
% ------------------------------------------------------------------------------
% ------------------------------------------------------------------------------

\newpage
\thispagestyle{empty}

%----------------------------------------------------------------------------------------
%	Projekte
%----------------------------------------------------------------------------------------
\topic{Eigene Software-Projekte}
\label{software_projects}

\NewEntry{Funktionale Programmierung}{Haskell}{\href{https://github.com/julmue/UntypedLambda}{Projektseite}}

\Description{\MarginText{Interpreter}%
Interpreter für das untypisierte Lambdakalkül; 
Die durch das untypisierte Lambdakalkül formalisierte Berechenbarkeit ist äquivalent zur Turing-Berechenbarkeit.
Weiter ist das System Grundlage für funktionale Programmiersprachen und findet Anwendung als 
Zwischensprache (\emph{Intermediate Language}) im Compilerbau.
}

\NewEntry{Funktionale Programmierung}{Haskell}{\href{https://github.com/julmue/lfo}{Projektseite}}

\Description{\MarginText{Interpreter}%
Interpreter für das einfach typisierte Lambdakalkül;
Dieses Kalkül ist das theoretische Fundament statisch typisierte funktionaler
Programmiersprachen und von zentraler Bedeutung in der Beweistheorie.
}

\NewEntry{Constraint Programmierung}{Haskell}{\href{https://github.com/julmue/Clank}{Projektseite}}

\Description{\MarginText{Solver}%
Solver für die parakonsistenten Aussagenlogiken K3, L3, LP, RM sowie für die 
klassische Aussagenlogik.
Diese Logiken werden unter anderem in der Robotik, künstlichen Intelligenz und 
Wissensrepräsentation angewendet.
}

\NewEntry{Constraint Programmierung}{Prolog}{\href{https://github.com/julmue/DpllDimacSatSolver}{Projektseite}}

\Description{\MarginText{SAT-Solver}%
Solver für die klassisch Aussagenlogik (Grundlage: DPLL-Algorithmus);\\
Industrielle Anwendungen findet dieses Verfahren bei der Lösung von
Baubarkeits- und Planungsproblemen sowie im Variantenmanagement.
}

\NewEntry{Beweistheorie}{Haskell}{\href{https://github.com/julmue/HaskellCurryHoward}{Projektseite}}

\Description{\MarginText{Beweistheorie /\\Kombinatoren-bibliothek}%
Der Curry-Howard-Lambek-Isomorphismus ist die zentrale Verbindung von 
funktionaler Programmierung, Logik und Kategorientheorie:
Programme sind konstruktive Beweise, konstruktive Beweise sind Morphismen.
}


%-------------------------------------------------------------------------------
\topic{Fremdsprachen}

\Description{\MarginText{Sehr gut}Englisch (\emph{verhandlungssicher})}
\Description{\MarginText{Grundlagen}Spanisch}


%-------------------------------------------------------------------------------
\topic{Sonstiges}

\NewEntry{08/2005-03/2006}{Work \& Travel}{Neuseeland}

\Description{\MarginText{Sprachreise\\Englisch}%
Sprach- und Arbeitsreise Neuseeland
}



\end{cv}

\end{document}
