\documentclass[noindent]{scrartcl}
\usepackage{../Header/header}

\usepackage[a4paper,width=13cm, height=27cm]{geometry}

\pagestyle{empty}

\setlist{nosep}

\begin{document}

\newcommand{\snp}{\href{www.snpgroup.com}{SNP SE}}
\newcommand{\herrenknecht}{\href{https://www.herrenknecht.com/}{Herrenknecht AG}}
\newcommand{\sdte}{\href{https://news.sap.com/germany/2019/05/arbeitsgruppe-migration-s4hana/}{Selective Data Transition Engagement}}

%----------------------------------------------------------------------------------------
%	NAME AND CONTACT INFORMATION SECTION
%----------------------------------------------------------------------------------------

\thispagestyle{empty}
\begin{tikzpicture}[remember picture,overlay]
    \node[align=center] at ([yshift=-7cm]current page.north)%
         {\huge \spacedallcaps{Curriculum Vitae}};
    % cover photo
    \node[align=center] (image) at ([yshift=-13cm]current page.north) %
         {\includegraphics[width=75mm]{Foto}};
    % name and address
    \node[align=center] (name) at ([yshift=-19cm]current page.north) %
         {\Huge \textcolor{magma}{\spacedallcaps{Julian Müller}}};
     \node[align=center] at ([yshift=-21.2cm]current page.north) %
         {\textcolor{magma}{\spacedallcaps{Logik, (M.A.)}}};
    \node[align=center] at ([yshift=-20.2cm]current page.north) %
         {\textcolor{magma}{\spacedallcaps{Maschinenbau, (B.E.)}}};
    \node[align=left] at ([xshift=0.3cm,yshift=-25cm]current page.north) % 
         {\color{textgray}
			\begin{tabular}{lr}
             Geburtsdatum & 29 Oktober 1984 \vspace{0.5em} \\

             Geburtsort & 77694 Kehl \vspace{0.5em} \\

             Email & \href{mailto:jul.mue@hotmail.de}{jul.mue@hotmail.de} \vspace{0.5em} \\

             % Website & \href{https://julmue.github.io}{julmue.github.io} \vspace{0.5em} \\

             Telefon & +49 176 20562999 \vspace{0.5em}  \\

             Adresse & Schlo\ss stra\ss e 34 \vspace{0.5em}  \\
                       & 14059 Berlin \\
            \end{tabular}
         };
\end{tikzpicture}

% Your name
\newpage

\thispagestyle{empty}

%\begin{cv}
\begin{cv}{\spacedallcaps{Julian Müller}}\vspace{1.5em} 

%----------------------------------------------------------------------------------------
%	WORK EXPERIENCE
%----------------------------------------------------------------------------------------

\topic{Berufspraxis}

% ------------------------------------------------------------------------------
% ------------------------------------------------------------------------------
% ------------------------------------------------------------------------------
\NewEntry{Seit 2020/01%
     \MarginText{Engineering Manager / Solution Owner R\&D}%
     }{Engineering Manager / Solution Owner R\&D}{\snp}

\Description{%
Design und Umsetzung eines Produkts zur Prozessautomatisierung basierend auf symbolischer KI, 
funktionaler Programmierung, ABAP und HANA.
Aufbau und Führung eines verteilten High-Preformance Teams zur Umsetzung des Produkts.
\begin{itemize}[label=\isym]
\item Engineering Manager: Aufbau und Führung eines 5-köpfigen verteilten Entwicklungsteams \details{proj_engineering_manager}
\item Solution Owner: Agiler Produktentwurf und Entwicklungssteuerung \details{proj_sdte}
\item Lead Engineer: Systementwurf, Architektur, Implementierung von Prototypen
\end{itemize}
Weiter:
\begin{itemize}[label=\isym]
\item Lead Engineer für die SNP im \sdte\ der SAP
\end{itemize}
}

\NewEntry{01/2019-01/2020%
     \MarginText{Engineer R\&D}%
     }{Engineer SAP Transformation R\&D}{\snp}

\Description{%
Entwicklung von Systemen für die selektive Datenmigration nach S/4 HANA.
\begin{itemize}[label=\isym]
\item Automatisierung der Finanz- und Logistikmigration nach S/4 HANA
\item Entwicklung von Analyse- und Tracingsystemen
\item Data Engineering und Data Analysis
\item Presales, Schulungen und Vorträge
\end{itemize}
Weiter:
\begin{itemize}[label=\isym]
\item Absolvent des SNP Exzellenzprogramms für junge Führungskräfte
\end{itemize}
}


\NewEntry{01/2017-01/2019\MarginText{Consultant SAP Transformation}%
     }{Consultant SAP Transformation}{\snp}

\Description{%
Entwicklung von Systemen für die  SAP Landscape Transformation (ERP und BW).
\begin{itemize}[label=\isym]
\item Entwicklung und Umsetzung von Transformationen für SAP BW
\item Finanz- und Logistikmigrationen
\item Data Engineering und Data Analysis
\item Presales, Schulungen und Vorträge
\end{itemize}
}

\NewEntry{10/2011-07/2012%
     \MarginText{Ingenieur R\&D}%
     }{Ingenieur R\&D (Werkstud.)}{\herrenknecht}

\Description{%
Anwendung von Verfahren der künstlichen Intelligenz zur Prozesssicherung im Tunnelvortrieb.
Entwurf, Programmierung und Inbetriebnahme eines 3D-Lasersystems zur Verleißmessung bei Abbauwerkzeugen.
}

\NewEntry{\MarginText{Ingenieur R\&D}09/2010-03/2011}{Ingenieur R\&D (Praktikum)}{\herrenknecht}

\NewEntry{01/2009-09/2009\MarginText{Tutor}}{Tutor Mathematik}{RFH Köln}

%----------------------------------------------------------------------------------------
%	EDUCATION
%----------------------------------------------------------------------------------------

\topic{Ausbildung}

\NewEntry{2013-2016\MarginText{\href{http://www.sozphil.uni-leipzig.de/cm/logik/}{Master Logik}\\
(Endnote: 1.6)}}{Logik (M.A.)}{Universität Leipzig}

\Description{%
{Abschlussarbeit (Note 1.0): \textit{Der untypisierte Lambdakalkül und seine Anwendung}%\
}}

%------------------------------------------------

\NewEntry{2008-2013%
     \MarginText{Maschinenbau\\(Endnote: 1.8)}%
     }{Maschinenbau (B.E.)}{RFH Köln}

\Description{%
Abschlussarbeit (Note 1.0): \textit{Verschlei\ss erkennung bei Werkzeugen im Tunnelbau}%
}

\NewEntry{2004-\MarginText{Event Organizer}}{Event Organizer}{Angell Institut Freiburg}

%-------------------------------------------------------------------------------
\topic{Sprachen}

\Description{\MarginText{Muttersprache}Deutsch}
\Description{\MarginText{Sehr gut}Englisch (\emph{verhandlungssicher})}
\Description{\MarginText{Grundlagen}Spanisch}


\newpage

%----------------------------------------------------------------------------------------
%	METHODOLOGY
%----------------------------------------------------------------------------------------
\topic{Methoden und Kenntnisse}

\NewMethodology{Team Development und Leadership}
\Description{\MarginText{Team Development und Leadership}

\begin{itemize}[label=\isym]
\item Aufbau und Führung verteilter Entwicklungsteams
\item Coaching Kommunikations- und Feedbackkultur
\item Coaching Methoden des Softwareentwurfs und der Softwarequalität
\item Einführung und Verbesserung agiler Entwicklungsmethoden
\item Koordination mit internern und externern Stakeholdern
\item Ressourcenplanung und -management
\end{itemize}
}


\NewMethodology{Agile Entwicklungsmethoden}
\Description{\MarginText{Agile Entwicklungsmethoden}

\begin{itemize}[label=\isym]
\item Scrum und Kanban
\item Design Thinking
\item Lean Management und Lean Startup
\end{itemize}
}

\NewMethodology{Projekterfahrung}
\Description{\MarginText{Projekterfahrung}

\begin{itemize}[label=\isym]
\item Engineering Manager - Aufbau und Führung eines Entwicklungsteams \details{proj_engineering_manager}
\item Solution Owner - Process Automation \details{proj_solution_owner}
\item Lead Engineer - SAP Selective Data Transition Engagement \details{proj_sdte}
\item Project Lead - Entwicklung einer Roadmap nach S/4 HANA \details{proj_roadmap_s4}
\item Engineer - Migration Rechnungswesen und Logistik nach S/4 HANA \details{proj_trafo_s4}
\item Engineer - Transformation SAP BW \details{proj_trafo_bw}
\end{itemize}
}

%----------------------------------------------------------------------------------------
%	TECHNOLOGY
%----------------------------------------------------------------------------------------
% \newpage

\topic{Technologien}

\NewTechnology{\MarginText{SAP Technologien}}{SAP Technologien}
\label{tec_sap}
\Description{

\begin{itemize}[label=\isym]
\item SAP ABAP
\item SAP HANA
\item SAP S/4 HANA, SAP R/3, SAP Netweaver Business Intelligence
\end{itemize}
}

\NewTechnology{\MarginText{Künstliche Intelligenz}}{Künstliche Intelligenz}
\label{tec_ai}

\Description{%
\begin{itemize}[label=\isym]
\item \href{https://en.wikipedia.org/wiki/Automated_planning_and_scheduling}{Automatisierte Planung}
\item \href{https://en.wikipedia.org/wiki/Boolean_satisfiability_problem}{Propositional Satisfiability (SAT)}
\item \href{https://en.wikipedia.org/wiki/Satisfiability_modulo_theories}{Satisfiability Modulo Theories (SMT)}
\item \href{https://en.wikipedia.org/wiki/Model_checking}{Bounded Model Checking}
\item \href{https://en.wikipedia.org/wiki/Artificial_neural_network}{Künstliche Neuronale Netzwerke (KNN)}
\item \href{https://en.wikipedia.org/wiki/Support_vector_machine}{Support-Vector-Machine (SVM)}
\end{itemize}
}


\NewTechnology{\MarginText{Data Science}}{Data Science und Data Engineering}
\label{tec_data_science}

\Description{%
Python, Haskell und R im Frontend sowie ABAP und HANA im Backend.
}

\NewTechnology{\MarginText{Infrastruktur}}{Infrastruktur, Continuous Integration und Deployment}
\label{tec_infrastructure}
\Description{

\begin{itemize}[label=\isym]
\item Virtualisierung: Hyper-V, Virtualbox
\item Versionierung und Continuous Integration: Git, Gitlab, abapGit
\item Containerisierung: Docker
\item Dependency Management: Make, Nix
\end{itemize}
}


\NewTechnology{\MarginText{Compilerbau}}{Compilerbau und Entwicklung von Programmiersprachen}
\label{tec_language}

\Description{%
Entwicklung von funktionalen und imperativen Programmiersprachen.
}


\NewTechnology{\MarginText{Programmiersprachen}}{Programmiersprachen}
\label{tec_languages}
\Description{

\begin{itemize}[label=\isym]
\item ABAP
\item Haskell
\item Python
\item R
\item SQL, SQLScript
\item JavaScript, TypeScript
\item Prolog
\item C
\end{itemize}
}


%----------------------------------------------------------------------------------------
%	Projects
%----------------------------------------------------------------------------------------
\newpage

\topic{Projekterfahrung}

\NewProject{Seit 2020/01\MarginText{Engineering Manager}}{Aufbau und Führung eines Innovation-Teams}{SNP}
\label{proj_engineering_manager}

\Description{%
Aufbau und Führung eines verteilten 5-köpfigen Entwicklungsteams; die Teamsprache ist Englisch.
Fokus liegt auf agilen Entwicklungsmethoden, der technologische Schwerpunkte sind SAP Landscape Transformationen, Compilerbau, funktionale Programmierung sowie Verfahren der symbolischen künstlichen Intelligenz.\newline
\textbf{Tätigkeiten als Engineering Manager}
\begin{itemize}[label=\isym]
\item Recruiting und Onboarding neuer Teammitglieder
\item Ressourcenplanung und -management für das Team
\item Definition und Kontrolle von Entwicklungs- und Qualitätszielen 
\item Implementierung agiler Entwicklungsmethoden
\item Individuelles Coaching und Teamentwicklung
\item Aufbau der Kommunikations- und Kollaborationskultur
\item Integration des Teams in das Unternehmen
\end{itemize}
}


\NewProject{Seit 2020/01\MarginText{Solution Owner}}{Produktentwicklung Process Automation}{SNP}
\label{proj_solution_owner}

\Description{%
Design und Umsetzung eines Produkts im Bereich Prozessautomatisierung basierend auf symbolischer KI, funktionaler Programmierung, ABAP und HANA. 
Das Aufgabenprofil umfasst dabei sowohl die Rolle des Solution Owners als auch die des Lead Engineers.\newline
\textbf{Tätigkeiten als Solution Owner}
\begin{itemize}[label=\isym]
\item Konzeption der Produktvision
\item Definition der Roadmap
\item Implementierung von Prototypen
\item Kommunikation und Abstimmung mit Stakeholdern und Experten
\item Pflege und Priorisierung des Backlogs
\item Sprintplanung und Abstimmung mit den Entwicklern während des Sprints
\item Koordination der Forschung
\end{itemize}
\textbf{Technologien}
\begin{itemize}[label=\isym]
\item Verfahren und Systeme der symbolischen künstlichen Intelligenz
\item Verfahren und Systeme des Compilerbaus und der funktionalen Programmierung
\item SAP ABAP
\item SAP HANA (Core Data Services, Graph Database, SQLScript)
\end{itemize}
}


\NewProject{Seit 2019/02\MarginText{Lead Engineer}}{SAP Selective Data Transition Engagement}{SAP}
\label{proj_sdte}

\Description{%
Das \sdte\ ist eine Arbeitsgruppe aus SAP, SNP, CBS, 
Natuvion und Datavard in welchem Konzepte, Technologien und Best-Practices erarbeitet werden um den Top 500 der ERP-Großkunden die Migration auf SAP S/4HANA zu ermöglichen.\newline
\textbf{Tätigkeiten als Lead Engineer}
\begin{itemize}[label=\isym]
\item Leitender Vertreter der SNP im Technology Stream des SDTE Projekts
\item Koordination mit anderen Streams (Governance, Architecture, Application)
\item Entwicklung einer graphenbasierten Wissensdatenbank zur Analyse von SAP-Konvertierungsmethoden (XPRA, XCLA, SDM, DDIC-Conversion-Exit)
     \begin{itemize}[label=\cdot]
	\item Anforderungen und Konzept
	\item Prototypenbau und Proof-Of-Concept
	\item Solution Owner für die Implementierung
     \end{itemize}
\end{itemize}	
\textbf{Technologien}
\begin{itemize}[label=\isym]
\item SAP Konvertierungstechniken: XPRA, XCLA, SDM, DDIC-Conversion-Exit
\item SAP HANA (Graph Database, SQLScript)
\item SAP ABAP
\end{itemize}
}

% Seitenumbruch
\newpage

\NewProject{2019/01-2019/10\MarginText{Project Lead / Lead Engineer}}{Roadmap nach S/4 HANA}{SNP}
\label{proj_roadmap_s4}

\Description{%
Im Rahmen des SNP Excellenzclusters für junge Führungskräfte wurde eine Roadmap nach S/4 HANA entwickelt. 
Basierend auf dem Design-Thinking-Prozess wurden in enger Zusammenarbeit 
mit den Stakeholdern diverse funktionale Prototypen erstellt.\newline
\textbf{Tätigkeiten als Project Lead / Lead Engineer}
\begin{itemize}[label=\isym]
\item Design Thinking: Planung und Durchführung von Workshops mit Wissensträgern und Stakeholdern
\item Design Thinking: Erstellen von Personas
\item Design Thinking: Entwicklung funktionaler Prototypen
\end{itemize}
\textbf{Technologien}
\begin{itemize}[label=\isym]
\item SAP ABAP
\item SAP HANA
\item Verfahren der symbolischen künstlichen Intelligenz
\item Verfahren des Compilerbaus und der funktionalen Programmierung
\end{itemize}
}


\NewProject{Seit 2019/01\MarginText{Engineer}}{Migration nach S/4 HANA}{SNP}
\label{proj_trafo_s4}

\Description{%
Der Aufbau und die Integration des Rechnungswesens und der Logistik 
sind kritische Elemente für eine Migration von R3 nach S/4 HANA.\newline
\textbf{Tätigkeiten als Engineer}
\begin{itemize}[label=\isym]
\item Integration von Migrationslogik in das SNP S/4 Cockpit
\item System- und Fehleranalyse
\item Projektsupport 
\item Ausarbeitung und Durchführung von Schulungen zur S/4 HANA Migration
\end{itemize}
\textbf{Technologien}
\begin{itemize}[label=\isym]
\item SAP ABAP
\item SAP HANA
\item SNP S/4 Cockpit
\end{itemize}
}


\NewProject{2017/07-2018/07\MarginText{Engineer}}{Konversion SAP BW}{Bosch, SNP}
\label{proj_trafo_bw}

\Description{%
Konversion einer aus fünf Systemen bestehenden BW-Landschaft basierend auf dem 
Regelwerk einer sich gleichzeitig in der Entwicklung befindlichen ERP-Migration.\newline
\textbf{Tätigkeiten als Engineer}
\begin{itemize}[label=\isym]
\item Technische Projektleitung eines 6-köpfigen Teams
\item Konzept und Umsetzung: Migration DSO InfoProvider
     \begin{itemize}[label=\cdot]
	\item Prototypenbau und Proof-Of-Concept
     \item Implementierung Full-Scope im Projekt
     \end{itemize}
\item Softwareentwicklung: Analysegetriebene Migration
     \begin{itemize}[label=\cdot]
	\item Analysebasierte Konfiguration der SNP Transformationswerkzeuge
	\item Sprachen: ABAP, R Language, Haskell
     \end{itemize}
\end{itemize}
\textbf{Technologien}
\begin{itemize}[label=\isym]
\item SAP BW
\item SNP TC
\item R Language
\item Haskell
\end{itemize}
}

\end{cv}

\end{document}
