\documentclass[noindent]{scrartcl}

\usepackage{header}

\begin{document}


%----------------------------------------------------------------------------------------
%	NAME AND CONTACT INFORMATION SECTION
%----------------------------------------------------------------------------------------

\thispagestyle{empty}
\begin{tikzpicture}[remember picture,overlay]
    \node[align=center] at ([yshift=-7cm]current page.north)%
         {\huge \spacedallcaps{Curriculum Vitae}};
    % cover photo
    \node[align=center] (image) at ([yshift=-13cm]current page.north) %
         {\includegraphics[width=75mm,height=105mm]{Foto}};
    % name and address
    \node[align=center] (name) at ([yshift=-19cm]current page.north) %
         {\Huge \textcolor{blaze}{\spacedallcaps{Julian Müller}}};
    \node[align=center] at ([yshift=-20.2cm]current page.north) %
         {\textcolor{blaze}{\spacedallcaps{Mechanical Engineering, (B.E.)}}};
    \node[align=center] at ([yshift=-21.2cm]current page.north) %
         {\textcolor{blaze}{\spacedallcaps{Logics, (M.A.)}}};
    \node[align=left] at ([xshift=0.3cm,yshift=-25cm]current page.north) % 
         {\color{textgray}
			\begin{tabular}{lr}
             Date of Birth & 29 Octobre 1984 \vspace{0.5em} \\

             Place of Birth & 77694 Kehl (Germany) \vspace{0.5em} \\

             Email & \href{mailto:jul.mue@hotmail.de}{jul.mue@hotmail.de} \vspace{0.5em} \\

             Website & \href{https://julmue.github.io}{julmue.github.io} \vspace{0.5em} \\

             Phone & +49 176 55509278 \vspace{0.5em}  \\

             Address & Josef-Gottwald-Stra\ss e 1 \vspace{0.5em}  \\
                       & 77654 Offenburg \vspace{0.5em} \\
                       & Germany\\
            \end{tabular}
         };
\end{tikzpicture}
%\thispagestyle{empty} % Stop the page count at the bottom of the first page

% Your name
\newpage
%\begin{cv}{\spacedallcaps{Julian Müller}}\vspace{1.5em} 

%----------------------------------------------------------------------------------------
%	EDUCATION
%----------------------------------------------------------------------------------------

\begin{cv}

\topic{Education}

\NewEntry{2013-2016}{Logic (M.A.)}{University of Leipzig}

\Description{\MarginText{\href{http://www.sozphil.uni-leipzig.de/cm/logik/}{Logics (Master)}\\
(Grade 1.6 /\\ GPA 3.4)}%
%Abschluss: 1.6 (angestrebt)\newline%
Specializations:
\EvenThree
    {\isym\ \textit{Constraint Programming}}%
    {\isym\ \textit{Paraconsistent Logics}}%
    {\isym\ \textit{Knowledge Representation}}
{Thesis (Grade 1.0 / GPA 4.0): \textit{The Untyped Lambda Calculus}\newline
\isym\ \textit{In Computer Science}: As a foundation for programming languages\newline
\isym\ \textit{In Proof Theory}: Curry-Howard-Lambek-isomorphism\newline
\isym\ \textit{In Mathematics}: As the internal language of cartesian closed categories\newline
Supervisor: Dr.~Peter \textsc{Steinacker}, Prof.~Thomas \textsc{Bartelborth}}\newline
Due: 14.07.2016}

%------------------------------------------------

\NewEntry{2008-2013}{Mechanical Engineering (B.E.)}{RFH Köln}

\Description{\MarginText{Mechanical Engineering (Bachelor)\\(Grade 1.8 /\\ GPA 3.2)}%
Specialization:
\EvenThree%
    {\isym\ \textit{Technical Optics / Laser Technology}}%
    {\isym\ \textit{Mechatronics}}%
    {\isym\ \textit{Programming in C}}%
Thesis (Grade 1.0 / GPA 4.0): \textit{Wear Detection of Cutting Tools in Tunneling}\newline
\isym\ \textit{Development and construction of a laser scanner for wear detection}\newline
\isym\ \textit{Comparison of methods for industrial image processing (2d/3d)}\newline
\isym\ \textit{Project specific analysis of geometric and logistic constraints of\\ 
               \hspace{0.8em}tunneling processes}\newline
Supervisor: Prof.~Werner \textsc{Simon}, Prof.~Marcus \textsc{Scholl}}

\NewEntry{2004-2007}{Event Organizer}{Angell Institut Freiburg}

\Description{\MarginText{Event Organizer}%
Training as an \textit{International Event Organizer}.}
%------------------------------------------------

%----------------------------------------------------------------------------------------
%	WORK EXPERIENCE
%----------------------------------------------------------------------------------------

\topic{Work Experience}

% ------------------------------------------------------------------------------
% ------------------------------------------------------------------------------
% ------------------------------------------------------------------------------
\NewEntry{04/2012-07/2012}{Working Student}{Herrenknecht AG}

\Description{\MarginText{Working Studend\\Research \& Development}%
Completion of the research project :\newline
Design and construction of a test bench for laser triangulation:\newline
\isym\ Calculation of the parameters of the optical measuring unit\newline
\isym\ Design, implementation and commissioning of the prototype
}

\NewEntry{10/2010-04/2012}{Diplomate (R\&D)}{Herrenknecht AG}

\Description{\MarginText{Diplomate\\Research \& Development}%
Comparison of procedurs for optical measurment:\newline
\isym\ Testing and examination of depth-map generating procedures\newline
\isym\ Examination of procedures for feature detection in image data\newline
\isym\ Analysis of processes in mechanized tunneling\newline
\isym\ Developing a conceptual design of an optical measuring system
}

\NewEntry{09/2010-03/2011}{Intern (R\&D)}{Herrenknecht AG}
     
\Description{\MarginText{Intern\\Research \& Development}%
Development of concepts for mechanized tunneling:\newline
\isym\ Project: concepts of mechanized tunneling in nuclear research\newline
\isym\ Concept for traffic tunneling: expansion of existing tunnels under \\
       \hspace{0.85em}upkeep of road- and railtraffic
}


\NewEntry{01/2009-09/2009}{Tutor (Mathematics)}{RFH Köln}

\Description{\MarginText{Tutor\\Mathematics}%
Tutorial for engineering mathematics:\newline
\isym\ Foundations of engineering mathematics\newline
\isym\ Calculus\newline
\isym\ Linear algebra
}

\NewEntry{04/2006-07/2006}{Intern (Administration)}{LLombart Export}
     
\Description{\MarginText{Intern\\(Administration)}%
Internship abroad --- task area in administration:\\ 
Office work, translations
}

% ------------------------------------------------------------------------------
% ------------------------------------------------------------------------------
% ------------------------------------------------------------------------------

\newpage
%----------------------------------------------------------------------------------------
%	Projekte
%----------------------------------------------------------------------------------------
\topic{Software Projects}

\NewEntry{Functional Programming}{Haskell}{\href{https://github.com/julmue/UntypedLambda}{Project Page}}

\Description{\MarginText{Interpreter}%
Interpreter for the untyped lambda calculus.
Besides being the paradigmatic language for functional programming, the lambda calculus sees 
wide application as an intermediate language for compilers.
}

\NewEntry{Functional Programming}{Haskell}{\href{https://github.com/julmue/lfo}{Project Page}}

\Description{\MarginText{Type-Checker}%
Interpreter for the simply typed lambda calculus.
This calculus is the theoretical foundation for statically typed functional programming languages
and of major importance in proof theory.
}

\NewEntry{Constraint Programmierung}{Haskell}{\href{https://github.com/julmue/Clank}{Project Page}}

\Description{\MarginText{Solver}%
Solver for the paraconsistent propositional logics K3, L3, LP, RM and for classic propositional calculus.
Applications for these  logics are in robotics, artificial intelligenze and knowledge representation.
}

\NewEntry{Constraint Programming}{Prolog}{\href{https://github.com/julmue/DpllDimacSatSolver}{Project Page}}

\Description{\MarginText{SAT-Solver}%
Solver for the satisfiablility problem (SAT) of classical propositional calculus;\\
This procedure is used widely in the industry,
especially for solving constructability and planning problems as well as in variant management. 
}

\NewEntry{Proof Theory}{Haskell}{\href{https://github.com/julmue/HaskellCurryHoward}{Project Page}}

\Description{\MarginText{Proof Theory /\\Combinator Library}%
The Curry-Howard-Lambek-isomorphism is the central connection 
of functional programming, logcis and category theory:
Programs are constructive proofs, and constructive proofs are morphisms.
}

%------------------------------------------------


%----------------------------------------------------------------------------------------
%	COMPUTER SKILLS
%----------------------------------------------------------------------------------------
\topic{Programming Languages}

\Description{\MarginText{Very Good} Haskell, Prolog}

\Description{\MarginText{Good} Java, C, MATLAB, Simulink, OCaml}

\Description{\MarginText{Basics} Scala, SQL, JavaScript/ECMAScript, HTML, CSS, C++, Bash, SMT-LIB}


\topic{Technologies}

\Description{\MarginText{Operating Systems} Linux (Ubuntu, Mint, \dots), Windows}

\Description{\MarginText{Version Control Systems} git}

\Description{\MarginText{Computer Vision} MathWorks Image Processing Toolbox (MATLAB)}

\Description{\MarginText{Testing Frameworks} XUnit-Frameworks in Java, Haskell, Prolog, C++, C, \dots}


%-------------------------------------------------------------------------------
\topic{Languages}

\Description{\MarginText{Mother Tongue} German}
\Description{\MarginText{Very Good} English}
\Description{\MarginText{Basics} Spanish}


%-------------------------------------------------------------------------------
\topic{Miscellaneous}

\NewEntry{08/2005-03/2006}{Work \& Travel}{New Zealand}

\Description{\MarginText{Work \& Travel}%
Language study- and working holiday New Zealand
}


\end{cv}

\end{document}
